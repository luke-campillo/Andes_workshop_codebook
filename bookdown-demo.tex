% Options for packages loaded elsewhere
\PassOptionsToPackage{unicode}{hyperref}
\PassOptionsToPackage{hyphens}{url}
%
\documentclass[
]{book}
\usepackage{amsmath,amssymb}
\usepackage{lmodern}
\usepackage{iftex}
\ifPDFTeX
  \usepackage[T1]{fontenc}
  \usepackage[utf8]{inputenc}
  \usepackage{textcomp} % provide euro and other symbols
\else % if luatex or xetex
  \usepackage{unicode-math}
  \defaultfontfeatures{Scale=MatchLowercase}
  \defaultfontfeatures[\rmfamily]{Ligatures=TeX,Scale=1}
\fi
% Use upquote if available, for straight quotes in verbatim environments
\IfFileExists{upquote.sty}{\usepackage{upquote}}{}
\IfFileExists{microtype.sty}{% use microtype if available
  \usepackage[]{microtype}
  \UseMicrotypeSet[protrusion]{basicmath} % disable protrusion for tt fonts
}{}
\makeatletter
\@ifundefined{KOMAClassName}{% if non-KOMA class
  \IfFileExists{parskip.sty}{%
    \usepackage{parskip}
  }{% else
    \setlength{\parindent}{0pt}
    \setlength{\parskip}{6pt plus 2pt minus 1pt}}
}{% if KOMA class
  \KOMAoptions{parskip=half}}
\makeatother
\usepackage{xcolor}
\usepackage{color}
\usepackage{fancyvrb}
\newcommand{\VerbBar}{|}
\newcommand{\VERB}{\Verb[commandchars=\\\{\}]}
\DefineVerbatimEnvironment{Highlighting}{Verbatim}{commandchars=\\\{\}}
% Add ',fontsize=\small' for more characters per line
\usepackage{framed}
\definecolor{shadecolor}{RGB}{248,248,248}
\newenvironment{Shaded}{\begin{snugshade}}{\end{snugshade}}
\newcommand{\AlertTok}[1]{\textcolor[rgb]{0.94,0.16,0.16}{#1}}
\newcommand{\AnnotationTok}[1]{\textcolor[rgb]{0.56,0.35,0.01}{\textbf{\textit{#1}}}}
\newcommand{\AttributeTok}[1]{\textcolor[rgb]{0.77,0.63,0.00}{#1}}
\newcommand{\BaseNTok}[1]{\textcolor[rgb]{0.00,0.00,0.81}{#1}}
\newcommand{\BuiltInTok}[1]{#1}
\newcommand{\CharTok}[1]{\textcolor[rgb]{0.31,0.60,0.02}{#1}}
\newcommand{\CommentTok}[1]{\textcolor[rgb]{0.56,0.35,0.01}{\textit{#1}}}
\newcommand{\CommentVarTok}[1]{\textcolor[rgb]{0.56,0.35,0.01}{\textbf{\textit{#1}}}}
\newcommand{\ConstantTok}[1]{\textcolor[rgb]{0.00,0.00,0.00}{#1}}
\newcommand{\ControlFlowTok}[1]{\textcolor[rgb]{0.13,0.29,0.53}{\textbf{#1}}}
\newcommand{\DataTypeTok}[1]{\textcolor[rgb]{0.13,0.29,0.53}{#1}}
\newcommand{\DecValTok}[1]{\textcolor[rgb]{0.00,0.00,0.81}{#1}}
\newcommand{\DocumentationTok}[1]{\textcolor[rgb]{0.56,0.35,0.01}{\textbf{\textit{#1}}}}
\newcommand{\ErrorTok}[1]{\textcolor[rgb]{0.64,0.00,0.00}{\textbf{#1}}}
\newcommand{\ExtensionTok}[1]{#1}
\newcommand{\FloatTok}[1]{\textcolor[rgb]{0.00,0.00,0.81}{#1}}
\newcommand{\FunctionTok}[1]{\textcolor[rgb]{0.00,0.00,0.00}{#1}}
\newcommand{\ImportTok}[1]{#1}
\newcommand{\InformationTok}[1]{\textcolor[rgb]{0.56,0.35,0.01}{\textbf{\textit{#1}}}}
\newcommand{\KeywordTok}[1]{\textcolor[rgb]{0.13,0.29,0.53}{\textbf{#1}}}
\newcommand{\NormalTok}[1]{#1}
\newcommand{\OperatorTok}[1]{\textcolor[rgb]{0.81,0.36,0.00}{\textbf{#1}}}
\newcommand{\OtherTok}[1]{\textcolor[rgb]{0.56,0.35,0.01}{#1}}
\newcommand{\PreprocessorTok}[1]{\textcolor[rgb]{0.56,0.35,0.01}{\textit{#1}}}
\newcommand{\RegionMarkerTok}[1]{#1}
\newcommand{\SpecialCharTok}[1]{\textcolor[rgb]{0.00,0.00,0.00}{#1}}
\newcommand{\SpecialStringTok}[1]{\textcolor[rgb]{0.31,0.60,0.02}{#1}}
\newcommand{\StringTok}[1]{\textcolor[rgb]{0.31,0.60,0.02}{#1}}
\newcommand{\VariableTok}[1]{\textcolor[rgb]{0.00,0.00,0.00}{#1}}
\newcommand{\VerbatimStringTok}[1]{\textcolor[rgb]{0.31,0.60,0.02}{#1}}
\newcommand{\WarningTok}[1]{\textcolor[rgb]{0.56,0.35,0.01}{\textbf{\textit{#1}}}}
\usepackage{longtable,booktabs,array}
\usepackage{calc} % for calculating minipage widths
% Correct order of tables after \paragraph or \subparagraph
\usepackage{etoolbox}
\makeatletter
\patchcmd\longtable{\par}{\if@noskipsec\mbox{}\fi\par}{}{}
\makeatother
% Allow footnotes in longtable head/foot
\IfFileExists{footnotehyper.sty}{\usepackage{footnotehyper}}{\usepackage{footnote}}
\makesavenoteenv{longtable}
\usepackage{graphicx}
\makeatletter
\def\maxwidth{\ifdim\Gin@nat@width>\linewidth\linewidth\else\Gin@nat@width\fi}
\def\maxheight{\ifdim\Gin@nat@height>\textheight\textheight\else\Gin@nat@height\fi}
\makeatother
% Scale images if necessary, so that they will not overflow the page
% margins by default, and it is still possible to overwrite the defaults
% using explicit options in \includegraphics[width, height, ...]{}
\setkeys{Gin}{width=\maxwidth,height=\maxheight,keepaspectratio}
% Set default figure placement to htbp
\makeatletter
\def\fps@figure{htbp}
\makeatother
\setlength{\emergencystretch}{3em} % prevent overfull lines
\providecommand{\tightlist}{%
  \setlength{\itemsep}{0pt}\setlength{\parskip}{0pt}}
\setcounter{secnumdepth}{5}
\usepackage{booktabs}
\usepackage{amsthm}
\makeatletter
\def\thm@space@setup{%
  \thm@preskip=8pt plus 2pt minus 4pt
  \thm@postskip=\thm@preskip
}
\makeatother
\ifLuaTeX
  \usepackage{selnolig}  % disable illegal ligatures
\fi
\usepackage[]{natbib}
\bibliographystyle{apalike}
\IfFileExists{bookmark.sty}{\usepackage{bookmark}}{\usepackage{hyperref}}
\IfFileExists{xurl.sty}{\usepackage{xurl}}{} % add URL line breaks if available
\urlstyle{same} % disable monospaced font for URLs
\hypersetup{
  pdftitle={Andes Phylogenomics Workshop},
  pdfauthor={Rachel Schwartz},
  hidelinks,
  pdfcreator={LaTeX via pandoc}}

\title{Andes Phylogenomics Workshop}
\author{Rachel Schwartz}
\date{2023-06-20}

\begin{document}
\maketitle

{
\setcounter{tocdepth}{1}
\tableofcontents
}
\hypertarget{getting-started}{%
\chapter{Getting Started}\label{getting-started}}

The evolutionary tree of life is fundamental to our understanding of the natural world.
This workshop will go through all of the steps of research in understanding the evolutionary history of several groups of Andean plants.
We will participate in field collections of plants and processing of samples for collections.
In real research projects, the species and collections should be considered carefully
with an expectation of the potential for phylogenetic results and evolutionary hypotheses.
Researchers should create voucher specimens to preserve the samples long-term,
as well as appropriate samples for DNA extraction.
Samples should be recorded and given a unique identifier, in order to connect the specimen to the sequencing data.

In this workshop we will not have an opportunity to experience the expected next step of
sample processing, DNA extraction, or sequencing.
Contemporary sequencing is often completed by an outside facility, although we will discuss options as part of this workshop.
We will continue with the analysis of sequence data,
particularly target-capture data from genome sequencing.
As part of this we will learn basic computing on the command line
and some approaches to writing code in the bash shell and R.
This will allow us to conduct data analysis that can apply to larger scale work than can be conducted on a laptop.
As part of the analysis of different datasets, we will see how in many cases we draw different conclusions from different datasets and how we examine phylogenies.
We will then use these phylogenies to conduct comparative analyses to understand how phylogenies inform our understanding of trait evolution.
Finally, we will ask you to present the results of your analyses.

\hypertarget{intro}{%
\chapter{Introduction}\label{intro}}

Syllabus of Wasta et al.~2020 \url{https://www.ncbi.nlm.nih.gov/pmc/articles/PMC7188297/\#pbio.3000667.s003} provides the following example:

\hypertarget{introduction}{%
\section{Introduction}\label{introduction}}

Biological research is turning to genetic research methods for a deeper look into the biological factors that encode various traits.
We can use genetic techniques to delimit species, define populations, understand reproductive patterns systems, and answer many other interesting biological questions.

This program is set in Colombia, where students will learn how field research is conducted, participate in sample collection, and then interpret genetic data from these species to examine their relationships and understand how to accurately conduct analyses in these areas.\\
This program will provide an introduction to next-generation sequencing technology to biologists, who will gain not only the skills requisite for field research but the technical know-how to employ genetic research tools.

\hypertarget{case-studies}{%
\section{Case Studies}\label{case-studies}}

In this course, we will focus on three specific cases

We will use data from species in the family Campanulaceae.
We will break into groups and each examine a subset of these data.
We will examine different datasets using different approaches to understand how individual datasets contribute to phylogeny estimation.

This tutorial steps through the basic process of analyses in evolutionary biology.
We use genome sequence data to build phylogenetic trees.
We then use these phylogenies and trait data to understand\ldots{}

\hypertarget{course-objectives}{%
\section{Course Objectives}\label{course-objectives}}

The goals of this course are to give participants advanced training in techniques important to the collection and analysis of plant evolution.

The course has the following broad objectives:

\begin{itemize}
\tightlist
\item
  To engage in both independent and team-based data collection
\item
  To teach sample collection techniques for plants
\item
  Learn to examine sequence data with code
\item
  Learn to build phylogenies from large datasets
\item
  Basic ability to look at other types of data
\item
  Be able to use comparative approaches to look at evolutionary questions based on phylogenies
\end{itemize}

\hypertarget{data-and-analysis}{%
\chapter{Data and analysis}\label{data-and-analysis}}

\hypertarget{genomic-data}{%
\section{Genomic data}\label{genomic-data}}

By definition genomes are large.
The human genome is 3.2 billion base pairs.
Plants can be even larger.
If you were to go through and compare two plant genomes by hand to discover the differences between them it would take a while.
To add to this complexity, genomes don't come off the sequencer as a 3.2 billion base pair sequence,
but as millions of small fragments.
Think of this sequencing method as tossing a book (your genome) into a paper shredder.
That means first you have to compare each fragment to a ``reference genome'' (like the real book)
and then figure out where your genome differs from the reference.
This is definitely a process you don't want to do by hand!

A computer (at least a big one with lots of computing power) can help us out by automating all these comparisons.
That means that before we can identify how the sequence of two plants differ
we need to build some skills to communicate with the server that we are using to work with our data.

\hypertarget{phylogenies}{%
\section{Phylogenies}\label{phylogenies}}

Different analyses produce different trees.
Different datasets produce different trees.
Rapid diversification makes analyses challenging.
Divergence dates can be estimated on trees with calibrations.

\hypertarget{comparative-analyses}{%
\section{Comparative analyses}\label{comparative-analyses}}

Understand models of character evolution and apply simple PCMs to existing datasets.
Calculating trait evolution.

\hypertarget{computational-skills}{%
\chapter{Computational skills}\label{computational-skills}}

The high-performance computer we'll use for our data doesn't have the graphical interface
you usually use (like a PC or Mac).
That means you have to type in your commands - you can't point and click.
This actually has a hidden benefit because it's easy to write down what you did in
a line of text rather than having to try to explain where to click on each step.

Let's start by learning how to access our server and run commands without clicking.

\hypertarget{log-in-to-the-server}{%
\section{Log in to the server}\label{log-in-to-the-server}}

For this duration of this workshop you will have access to a server at the University of Rhode Island. Go to \texttt{rstudio.uri.edu} and log in with the first part of your email address as both the username and password.

\hypertarget{load-and-set-up-data}{%
\section{Load and set up data}\label{load-and-set-up-data}}

We will be using a dataset to practice that will look a lot like the data you plan to collect.
By the end of this workshop you should be able to examine your own data in a similar way.
Our practice dataset comes from Onstein et al.~2019, which you can find at \url{https://onlinelibrary.wiley.com/doi/10.1111/jbi.13552}.
We have uploaded this paper to the server you are working on and we will show you how to access it there shortly.

Let's start by getting organized.
We will use both R code and bash/shell scripts in this workshop.
Because RStudio provides a mechanism for setting up projects we will use this to stay organized.
In the RStudio File menu select New Project - New Directory - New Project.

\begin{itemize}
\tightlist
\item
  Give the project a name (e.g.~evolution\_workshop). For this and all future work ensure that names never include spaces.
\item
  Select R 4.1.2
\item
  Create Project
\end{itemize}

By creating a new project you have created a folder for your work.
When using projects whenever you come back to this project you will return to the same setup.
This allows you to switch projects with different folders and open files.

\hypertarget{analyses}{%
\section{Analyses}\label{analyses}}

Begin your work by making a script.
This is where you will keep track of all of your analytical commands.
From the File menu select New File - R script.

To analyze data in \texttt{R} we need to load some helper ``libraries''. A common library for analysis is the \texttt{tidyverse}, which wraps multiple libraries made by RStudio. For details and cheatsheets see \url{https://www.tidyverse.org/} . We will also use a library that allows us to read Excel files.

Put the following in your script then click the Run icon.

\begin{verbatim}
library(readxl)
library(tidyverse)
\end{verbatim}

Now we will write a command to read the Excel file with the data from the paper.
The command we use is \texttt{read\_xlsx}.
This command takes two arguments in parentheses.
The first is the ``path'' to the data.
We will discuss paths more extensively later.
For now use the provided argument.
The second argument is the Excel sheet or tab we want to read.
Additionally, when we read the file into memory we assign the information to a ``variable''.
You can think of the variable as a box to contain the information.

\begin{verbatim}
onstein <- read_xlsx('../../shared/AndesWorkshop2023/Onstein_data.xlsx',
                     sheet = 'Matrix for analysis')
\end{verbatim}

When you run this command you will see the data saved in the environment.
When you click on the variable in the environment you will be able to view the data in tabular format.

Before we can do anything with the data, we need to make sure that the computer understands
the data the way we expect.
Click on the arrow next to the variable name in the environment.
Note that all of the columns of data are listed as ``chr''.
This means that the information is viewed as characters.
Some of our data is actually numeric and we must convert it in order to analyze it correctly.

To view the column names we use the \texttt{colnames} command and provide the dataset as the argument.

We access a single column by specifying the dataset name, followed by a \texttt{\$} followed by the column name. For example:

\begin{verbatim}
onstein$Log_Fruit_length_avg
\end{verbatim}

We assign a ``corrected'' version of the data to this variable.
For example:

\begin{verbatim}
onstein$Log_Fruit_length_avg <- as.numeric(onstein$Log_Fruit_length_avg)
\end{verbatim}

Repeat this analysis to ensure that all columns match your expectations.

Now we can visualize the data.
We use the command \texttt{ggplot} and provide the data and the independent and dependent variables we expect on the graph.
For example, we can examine the relationship between seed length and width.

\begin{verbatim}
ggplot(onstein, aes(Log_Seed_width_avg, Log_Seed_length_avg))
\end{verbatim}

Note that the variables are provided in an extra function \texttt{aes}.

If you run this command it will generate an empty plot. To plot the points you have to ``add'' a ``layer'' on this base of the plot style. In this case we use the \texttt{geom\_point} function to add the data as a scatter plot.

\begin{verbatim}
ggplot(onstein, aes(Log_Seed_width_avg, Log_Seed_length_avg)) + geom_point()
\end{verbatim}

Do you notice an apparent relationship between these variables?
We can view this more clearly by \emph{adding} a regression line to our plot.

\begin{verbatim}
geom_smooth(method = "lm", se = FALSE)
\end{verbatim}

The paper from which these data are drawn found ``syndromes'' in frugivory-related traits
such that fruits and seeds with dispersal by particular mechanisms (e.g.~mammals, birds, and bats) have common trait values.
Thus, we can see relationships among traits as above.
The paper describes:

\begin{itemize}
\tightlist
\item
  ``mammal syndromes of few, syncarpous fruits with many seeds, large fruits, and large seeds''
\item
  ``bird syndromes of many, bright-coloured, small fruits with few, small seeds and long stipes''
\item
  ``bird syndromes of dehiscent fruits with small seeds''
\item
  ``bat syndromes of dull-coloured, cauliflorous fruits''
\end{itemize}

Try plotting some relationship on your own. To get you started, you can plot \texttt{Log\_Fruit\_length\_avg} v \texttt{Log\_Seed\_number\_avg} and \texttt{Log\_Fruit\_length\_avg} v. \texttt{Log\_Seed\_length\_avg}. Try others as well.

If you tried plotting with a character variable on the x axis you might find the results difficult to look at.
You can try using a boxplot with \texttt{geom\_boxplot()} or other plot type instead.
Check out the ggplot cheatsheet for ideas: \url{https://posit.co/resources/cheatsheets/?type=posit-cheatsheets\&_page=2/}

When you have several interesting relationships it's advisable to confirm that the relationship is statistically significant.
We can examine relationships with a linear model using the \texttt{lm} function.
For example:

\begin{verbatim}
lm(Log_Seed_length_avg ~ Log_Seed_width_avg, data = onstein)
\end{verbatim}

Note that the format of the arguments is slightly different than in a \texttt{ggplot}.

The output of this function does not provide information without additional work.
Assign the output to a new variable.
Then provide this variable as the argument to the \texttt{summary} command.
Examine the p and r-squared value.
Do these match your observation from the graph?

Note: make sure your variable names are informative.

Repeat this process for your other graphs. Make sure you understand the biological interpretation of your results. Discuss with you instructor as needed.

\hypertarget{computational-skills-1}{%
\chapter{Computational skills}\label{computational-skills-1}}

\hypertarget{basic-navigation}{%
\section{Basic navigation}\label{basic-navigation}}

Click on the \texttt{Terminal} tab to access the command line interface, which we'll be using for our genomic analyses.

To see the list of files on this computer in your ``home'' directory do the following:

\begin{verbatim}
ls
\end{verbatim}

This folder might be empty because you haven't put anything here.

We will begin our work using the command line or shell.
Select the Terminal tab.
First, make a directory to put your scripts in inside your workshop folder.
We'll write these scripts in a little while, but basically a script is a list of commands that you will give the computer to do all the steps to analyze your data.

Start by checking which folder you are in using the following command.

\begin{verbatim}
pwd
\end{verbatim}

If you are already in your project folder you may skip the next step.
Otherwise you must move into that directory.
To change directories you need two things - the command to change directory and the argument that specifies which directory to move into.

\begin{verbatim}
cd evolution_workshop
\end{verbatim}

To make a directory you again need a command (make directory) and an argument (the name).

\begin{verbatim}
mkdir scripts 
\end{verbatim}

Now we can move into that folder.
\texttt{cd} means change directory so we'll use that a lot to open different folders.
This command also has an argument.

\begin{verbatim}
cd scripts 
\end{verbatim}

Now to get back to your home folder you can't \texttt{cd} and give it a folder name because
the computer will look in your current folder (scripts) for another folder.
Instead you need to tell the computer to ``move one level up'', going outside the current folder to the one containing it.

\begin{verbatim}
cd ..
\end{verbatim}

To practice, repeat this process but make a directory called \texttt{results}.
Now list the files in your current home directory again.
Do you see the two folders you just made?

If you want to check out what is in these folders (they are currently empty but we will add to them later)
you have two options.
First, you can change directory with the \texttt{cd} command and list the contents of the folder with \texttt{ls}.
Think about these folder just like a filing cabinet.
You started off in your home folder and now you are opening the results folder.

Alternatively, list the contents of a folder by given \texttt{ls} the folder name as an argument (e.g.~\texttt{ls\ scripts}).

We will learn lots more commands as we work through our data.

Adapted from \url{http://swcarpentry.github.io/shell-novice/}

\hypertarget{getting-your-data}{%
\chapter{Getting your data}\label{getting-your-data}}

For this workshop we will be using genome sequence data from a group of Andean plants.
For starters we will take a peak at these data so you know what it looks like before we use standard programs to analyze it.

The sequence data is in a shared folder on this server.
Relative to your workshop folder you can find the data at
\texttt{../../shared/AndesWorkshop2023/}.

If you cd to that directory and list the contents you can see folders containing files of data.

Sequence files for samples can be found in different folders labeled by taxon set.
Start by going to the folder for your assigned taxon set and listing its contents.
You should see two folders.
Go into the one for your assigned dataset.

You generally do not want to view sequence data files because they can be large and are rather hard to read.
If you want to see how large use \texttt{ls\ -lh}.
You are already familiar with the \texttt{ls} command for listing the contents of folders.
The dash followed by additional letters are called flags.
The \texttt{l} flag indicates that we are listing the contents in long form to provide additional information.
The \texttt{h} flag means we'll view the data in human readable format (i.e.~with size given in Kb or Mb).

Raw data is in the form of zipped fastq files, which are generated by a sequencer.
You can tell this by the very last bit of the name of the files, which is .fastq.gz or .fq.gz (just like you might see Word files labeled as .docx).
Just as a note, you often need to do some cleanup of the data that comes of the sequencer
so you only have high quality data.
We have done this step for you already.
Take a look at a little bit of one file.
The \texttt{less} command lets you scroll through the (very large) file.
\texttt{zless} is the \texttt{less} command that works on zipped files.

Use the following command replacing {[}{]} with a specific filename.

\begin{verbatim}
zless [].fastq.gz
\end{verbatim}

Fastq files contain sets of four lines.
The first line is the name of the sequence.
The second line is the sequence.
The fourth line is the quality of the sequence.
As you scroll through you should see many sets of four (one for each sequence).

Once you are done scrolling use q (for quit) to get back to your prompt.

Before we look at our large files we are going to practice on a small example file.
Move back to the main \texttt{AndesWorkshop2023} folder.
List the contents and notice the \texttt{example.fq} file.
For this example I have unzipped it.
You can view the entire contents of the file using the command \texttt{cat}.

\begin{verbatim}
cat example.fq
\end{verbatim}

Alternatively you can use the command \texttt{head} to view the first few lines of the file.

\begin{verbatim}
head example.fq
\end{verbatim}

We can also get a sense of the amount of data by counting the number of lines using the word count command \texttt{wc}.

\begin{verbatim}
wc example.fq
\end{verbatim}

This command tells you the number of lines, words, and characters. If you want to focus just on the number of lines you should use the \texttt{l} flag.

\begin{verbatim}
wc -l example.fq
\end{verbatim}

If we only want to get the names of the sequences we can search for lines that match a particular pattern and print just those. We search with the command \texttt{grep}. The pattern we are searching for is lines that start with @. And then we provide the name of the file.

\begin{verbatim}
grep '^@' example.fq
\end{verbatim}

Because this is a small example file we are able to print all the header lines without it being overwhelming. However, with our larger files we don't want to do this. However, we know that the \texttt{head} command allows us to view just the first few lines of our output. Now we'll combine those two commands using a pipe (\texttt{\textbar{}}). First we enter the \texttt{grep} command (with it's two arguments) then we ``pipe'' the output to the \texttt{head} command rather than printing it to the screen.

\begin{verbatim}
grep '^@' example.fq |  head
\end{verbatim}

Notice that the \texttt{head} command does not have arguments in this case because it accepts the output from the grep command.

The \texttt{head} command accepts a flag with the number of lines you want to print.
For example \texttt{head\ -4} will print the first four lines.

Your first challenge in this course is to combine the information that you have learned to get the number of header lines in one of your zipped fastq files (using \texttt{zgrep}).

\hypertarget{repeating-analyses}{%
\section{Repeating analyses}\label{repeating-analyses}}

One approach to analyzing multiple files in the same way is to copy your code and edit each copy to include the name of a particular file. That might be feasible (if slow) for the 10 files you are working with here, but it is inconvenient for the many files you might work with on a larger genomics project. Instead, we can make a list of the files we want to work with and tell the computer to repeat the analysis on each file.

In the prior challenge you figured out what analysis you want to do (list the number of header lines). Now we need to learn how to repeat this. In most programming languages, including our command line, we use a loop to go through a list and repeat that command. On the command line the format to do this looks like

\begin{verbatim}
for thing in list_of_things
do
    command $thing
done
\end{verbatim}

We start with \texttt{for} to indicate that we are about to make a loop.
Each time the loop runs (called an iteration), an item in the list is assigned in sequence to the variable \texttt{thing}, and the commands inside the loop are executed.
Think of a variable as a box that holds the item we are working with.
For example, if we are working with a list of files then the first time
through the loop \texttt{thing} is the first file name in the list,
the second time through \texttt{thing} is the second name, and so on.
This allows us to repeat our command for each item in the list using the same variable name.
Inside the loop, we indicate the variable by putting \texttt{\$} in front of it.
The \texttt{\$} tells the shell interpreter to treat the variable as a variable name and substitute its value in its place, rather than treat it as text or an external command.

\begin{itemize}
\tightlist
\item
  Start by figuring out what goes on the third line of the loop
\item
  Now figure out what goes on the first line (i.e.~your list)
\item
  When you run your loop it should print the number of header lines for each file
\end{itemize}

We might want to add one command to this loop to print out the name of the file prior to printing the number of header lines so we can more easily keep track of each file's information. The \texttt{echo} command will print whatever comes after it. Try inserting this command into your loop in order to print the name of the file following by the number of header lines.

Did you need to list all your files by hand? Copying and pasting a lot of names can be time consuming and error prone. Logically, you want to make your list following a particular rule, for example all of the files in this folder. We can use a wildcard (denoted \texttt{*}) to do this. So instead of writing out all the file names, replace all of that just with \texttt{*} as your list.
In other situations keep in mind that we might want to be more precise.
\texttt{*} really suggests any number of any character.
To indicate all files that end in \texttt{fastq.gz} we could use \texttt{*fastq.gz} as our list.

\hypertarget{tree-thinking-quiz}{%
\chapter{Tree thinking quiz}\label{tree-thinking-quiz}}

Before we delve into trees it's good to see what you are thinking so you can understand (at least partly) what you learn later.

\hypertarget{target-capture-data}{%
\chapter{Target capture data}\label{target-capture-data}}

\hypertarget{background}{%
\section{Background}\label{background}}

Much of the information in this chapter is drawn from the HybPiper tutorial, which can be found at \url{https://github.com/mossmatters/HybPiper/wiki/Tutorial}

First make a folder in your home directory to hold your outputs for this component of the project.

We have pre-installed the program HybPiper, which assembles target sequences from short high-throughput DNA sequencing reads.
By assembling each target sequence for each species we will be able to produce a dataset that allows us to understand the evolutionary history of the species.

To access HybPiper run.
This activates a conda environment (don't worry about this!).

\begin{verbatim}
conda activate /mnt/homes4celsrs/shared/envs/hybpiper
\end{verbatim}

In this case, our target sequences are the Angiosperm 353 standard targets.
Target capture is a standard technique for sequencing many known loci simultaneously using a library of sequences.
We have provided you with low-coverage sequence data from using these target sequences as probes (your fastq files).
HybPiper first assigns each read to a particular target using alignment software (BWA).
These reads are then assembled using a standard genome assembler (Spades).
HybPiper outputs a fasta file of the (in frame) CDS portion of each sample for each target region.

\hypertarget{running-hybpiper-to-assemble-genes}{%
\section{Running HybPiper to assemble genes}\label{running-hybpiper-to-assemble-genes}}

The basic command for HybPiper is as follows:

\begin{verbatim}
hybpiper assemble -t_dna targets.fasta -r species1_R*_test.fastq --prefix species1 --bwa --cpu 3
\end{verbatim}

Notice the commands and flags.
Move to the results directory in your workshop directory to run the following commands.
You should run this command, with the following changes:

\begin{itemize}
\tightlist
\item
  The argument for t\_dna is the fasta file in the shared workshop folder - you will need to indicate a full relative path not just the file name
\item
  The argument for the the r flag is the filename(s) including paths for the fastq files of interest. Note the * indicates that we will use both pairs of the sequencing read files.
\item
  The prefix should be indicative of the species name
\item
  Note that we are sharing a server so we are specifying 3 cpu's per group - if you are running on your own machine you can omit this flag and HybPiper will use all available resources. With limited shared resources this run may take a few minutes.
\end{itemize}

To view your results, list the contents of the folder and folders inside this.
You should see a folder for each gene.
Within each folder you can see several fasta files, which you can view.

To automatically run all samples consecutively you will need to loop through them.
Go back and look at the prior example of loops.
One approach is to make a list of all the names of the species in a file and then read that and use them one at a time.

First we need to make the list. We want to automate this process because in some cases you could have a lot of samples. Additionally, copying and pasting names into a list is prone to error.
Additionally, once you automate this process you could repeat it easily and quickly for other datasets.
For starters make a list of just the R1 files (you don't want R2 files because they have the same name so you would have duplicates).
First cd back to your data folder.
Output your list to a file in your results folder named \texttt{namelist.txt} by ``redirecting'' the output using \texttt{\textgreater{}}.

\begin{verbatim}
ls *R1* > ~/evolution_workshop/results/namelist.txt 
\end{verbatim}

Check that this worked using \texttt{cat} or by viewing the file and counting the number of lines..
We will need to adjust this output do remove the R1 label so that we can use wildcards to list both files together.
One approach is to notice a pattern in filenames: the files start with a sample identifier followed by an underscore followed by another identifier.
Thus, we can ``cut off'' all of the name after the second underscore and still have a unique name for our sample.
We use the ``cut'' command to make our data into columns using \_ (specify a delimiter with -d), then we select the first two of these (use the field flag -f).

\begin{verbatim}
ls *R1* | cut -d '_' -f 1,2 > ~/evolution_workshop/results/namelist.txt 
\end{verbatim}

Check this worked.
Now we can run HybPiper on all of our data sequentially.
In the following example, the species names are entered in the namelist.txt file.
The loop will iterate through them one at a time and run HybPiper.

\begin{verbatim}
while read name
do 
    [insert command here]
done < [path to]/namelist.txt
\end{verbatim}

This analysis will take some time. Check with an instructor before running this command to ensure you have done everything correctly.
Additionally, it's advisable to run this command in the background.
That means the command will run but you will continue to have access to your prompt so you can work.
Additionally you can log out of the server (eg for lunch) and the command will continue to run.
To run the command in background enclose the entire command in \texttt{\{\ \}}
and then add \texttt{\&\textgreater{}\ output.log\ \&} at the end to write the information that would normally print to the screen (as you saw when running a single command) to the file \texttt{output.log}.
You can view this file to watch the progress of your command.

\hypertarget{process-outputs}{%
\section{Process outputs}\label{process-outputs}}

To obtain some information about the results of this process use the \texttt{hybpiper\ stats} command as follows.

\begin{verbatim}
hybpiper stats -t_dna [target file] gene [path to]/namelist.txt
\end{verbatim}

From \url{https://hackmd.io/@mossmatters/HkPM7pwEK\#Getting-HybPiper-Stats}

Hybpiper stats will generate two files, seq\_lengths.tsv and hybpiper\_stats.tsv.
The first line of seq\_lengths.tsv has the names of each gene.
The second line has the length of the target gene, averaged over each ``source'' for that gene.
The rest of the lines are the length of the sequence recovered by HybPiper for each gene. If there was no sequence for a gene, a 0 is entered.

\hypertarget{viewing-information}{%
\subsection{Viewing information}\label{viewing-information}}

While you can \texttt{cat} or \texttt{head} this file it is difficult to view this information on the command line.
However, R is excellent for reading data in this format.

\begin{itemize}
\tightlist
\item
  Click the Console tab
\item
  While you could run commands here as you have in the shell, instead we will create a script to keep track of all of your commands. Note that this is also possible in the shell.
\item
  Select File - New File - R script
\item
  Save this file in the scripts folder
\item
  Load the tidyverse library
\end{itemize}

We will use the \texttt{read.table} command because our data are plain text separated by tabs.
We have three arguments: the filename (including path), how the columns are separated, and whether the file has a header line.

Note: R assumes you are in your project folder so all paths should be relative to this.

\begin{verbatim}
stats <- read.table("results/hybpiper_stats.tsv", sep = "\t", header = TRUE)
\end{verbatim}

View your stats data.

Repeat this analysis for \texttt{seq\_lengths.tsv}.

Given your observation of the output tables, you may be interested in some of the following questions:

\begin{itemize}
\tightlist
\item
  What are the min, max, and average number of genes retrieved across samples?
\item
  What is the range of lengths retrieved for a given gene?
\end{itemize}

You should take a look at the data and imagine how you would calculate this.
Unfortunately,this process (e.g.~calculating the min value of each column individually) could be challenging to communicate to the computer effectively.
Furthermore, when we look at data in this ``rectangular'' form, we often want to ensure that our
data are ``tidy''.
Tidy data usually has one sample per row.
Currently our data have many observations per gene per row.
If we were to look at the lengths file and make a new row that includes our
calculation of the standard deviation of the gene lengths this would be a row with summary
information not sample information.

Instead of working with these data frames directly we are going to take a look at an example
that I have made for you that includes stats for just two samples in ``long form.''

\begin{itemize}
\tightlist
\item
  Read the file in the shared workshop folder using the \texttt{read\_csv} command.
\item
  Take a look at how these data are organized. Can you see how the data are organized that each row contains one piece of information?
\item
  Plot \texttt{value} v. \texttt{stat} as a scatter plot.
\end{itemize}

We can add a couple of tweaks to make our view better. First we can specify that we want each sample to be a different color by including a \texttt{color\ =\ Name} argument in our \texttt{aes}. Second we can rotate the graph 90 degrees to better view the data and labels by adding a layer \texttt{coord\_flip()}.

We can also generate a summary table with one row per stat and information in that row including the mean, standard deviation, etc. across genes.

The first component of this is to develop groups of rows by stat.
I like to imagine this as drawing boxes around all rows that contain a particular stat.
We use the \texttt{group\_by} function to communicate to the computer these ``boxes''.

\begin{verbatim}
group_by(stats_example, stat)
\end{verbatim}

If you run this command the data won't appear to be any different.
Now we need to generate a summary table where each group is collapsed into a single row in our new table containing the stat, its average, and standard deviation.

\begin{verbatim}
mean_stats <- group_by(stats_example, stat) %>% summarize(mean_across_species = mean(value),
                                                       sd = sd(value))
\end{verbatim}

Note that here we use \texttt{\%\textgreater{}\%} as the pipe command to send the output of one command to the next.
You have send the shell pipe \texttt{\textbar{}} and this works the same way. In our summarize command we provide
new column labels and what the contents of these columns will be.

Click on your \texttt{mean\_stats} variable to view your summary table.

In order to work with your lengths table you need to know how to produce a long format table.
This is tricky and will take some practice so don't worry if it seems complicated initially.
We will use the \texttt{pivot\_longer} function.
If you haven't noticed already, when you type a function and hit the tab button on your keyboard you will see a list of arguments.
The first argument you need is the data.
The second argument is the list of columns that will not be in the long table and currently contain the observation data.
For us this is all the columns from NumReads to GenesWithChimeraWarning.
Now we need to envision our new table. Go back and look at the example table for some help here.
I specified a table with a column to indicate the particular stat we are measuring and
a second column for the actual value observed.
This command, with these four arguments looks like the following:

\begin{verbatim}
stats_long <- pivot_longer(stats, cols = NumReads:GenesWithChimeraWarning, names_to = 'stat', values_to = 'value')
\end{verbatim}

Now try to work with the lengths table. First make a long format table. You want to output a table with the columns Species, gene, and length.

Did you notice that the mean lengths are already included in this table? That's really useful but also these data are not information about each sample and if we want to make summary tables and graphs we don't want to include them.

We can use the \texttt{filter} command to only include data that doesn't specify MeanLength in the Species column.

\begin{verbatim}
lengths_long_filtered <- filter(lengths_long, Species != "MeanLength")
\end{verbatim}

Now can you filter this output to include only data where the value in the length column is greater than 0.

\begin{itemize}
\tightlist
\item
  Now you should be able to make a summary table including the the mean length per gene, standard deviation of length per gene, and count of genes. Note that we did our second filtering step so these values would be accurate. Additionally, the counting function is \texttt{n()} (no arguments required).
\end{itemize}

Save your script!

\hypertarget{obtain-gene-data}{%
\subsection{Obtain gene data}\label{obtain-gene-data}}

At this point, we will go back to our Terminal.
Remind yourself of your folders and data and note each sequence is in its own folder.
We can use HybPiper to fetch the sequences recovered for the same gene from multiple samples and generate a file for each gene. Use the following command:

\begin{verbatim}
hybpiper retrieve_sequences dna -t_dna [target file]  --sample_names [path to]/namelist.txt
\end{verbatim}

You should now see one file per gene in your results folder.
Each file is fasta formatted with the data for each available species.
In the next section we will create an alignment of all species for each gene.

Note that HybPiper has additional features we will not use in this workshop due to a lack of time.

\hypertarget{visual-inspection-of-data}{%
\section{Visual inspection of data}\label{visual-inspection-of-data}}

To check that your genes assembled correctly and are likely the correct species you may use BLAST, which is found at \url{https://blast.ncbi.nlm.nih.gov/Blast.cgi}

\begin{itemize}
\tightlist
\item
  Select Nucleotide BLAST
\item
  Paste one of the contigs from one of the genes from one of the species
\item
  Select Somewhat similar sequences (blastn)
\item
  BLAST
\item
  Examine the species and the match
\end{itemize}

\hypertarget{results}{%
\chapter{Results}\label{results}}

\hypertarget{alignment}{%
\section{Alignment}\label{alignment}}

In order to use these data to build our phylogenies, we need to align each gene.
This allows our tree-building software to compare species using nucleotides that we believe share ancestry.
We will use the software MAFFT.
Due to the nature of this server different programs are in different ``environments'' so you'll need to deactivate the hybpiper environment and activate one for mafft.

\begin{verbatim}
conda deactivate
conda activate /mnt/homes4celsrs/shared/envs/mafft
\end{verbatim}

Now MAFFT is available to run.
However, before we align our data let's take a look at the output.
Do you remember how to print out particular lines of a file?
Try printing all of the lines that start with \texttt{\textquotesingle{}\textgreater{}\textquotesingle{}} in a particular FNA file. \textbf{Make sure to use \texttt{\textquotesingle{}\textgreater{}\textquotesingle{}} including the single quotes when searching.}
You should notice that HybPiper has added some information to these lines so they include more than just the name of the sample. Because these additions differ across genes subsequent analyses may treat them as separate samples.

Before we do the alignment we need to remove this extra information.
Again, we can think about the pattern we are looking for: we want to go through each FNA file (this should suggest using a loop) and keep the first ``word'' in each line (think of a word as a set of characters not separated by spaces).
We use the \texttt{cut} command to divide each line of our file into columns.
We use the \texttt{-f} flag to indicate the field (column) we want to keep, and the \texttt{-d} flag to indicate how to separate columns (i.e.~with a space: \texttt{\textquotesingle{}\ \textquotesingle{}}).

An example cut command for one file could look like the following:

\begin{verbatim}
cut -f 1 -d ' ' 7577.FNA > 7577.FNA.fa
\end{verbatim}

Now use a loop as before to repeat this command for all FNA files.

Now we can loop through each of these files and output an alignment.
The basic mafft command looks like the following, assuming you replace FILE with a particular file.

\begin{verbatim}
mafft --auto --thread 1 7577.FNA.fa > 7577.FNA.fa.fasta
\end{verbatim}

Remember you have 350 genes (i.e.~350 fasta files) so you want to run these alignments in a loop.
As before, write a loop to go through each FNA.fa file and output an alignment.

\hypertarget{view-in-r}{%
\section{View in R}\label{view-in-r}}

The best way to view an alignment is in a specialized program, but because we have R
easily available we'll view an example here.

Make a new script to run this alignment view.
You should first load the \texttt{ape} and \texttt{ggmsa} library. (This uses the gg as in ggplot and msa, which stands for multiple sequence alignment.)

\begin{itemize}
\tightlist
\item
  Read in your fasta file with ape's \texttt{read.FASTA}.
\item
  Use the ggmsa command as follows, substituting the particular variable that contains your fasta file. Allow a little while for this to run. You may need to click the Zoom button above the plot to see it more clearly.
\end{itemize}

\begin{verbatim}
ggmsa(seqs_7577, start = 20, end = 120, char_width = 0.5, seq_name = T, color = "Chemistry_NT") +
  geom_msaBar()
\end{verbatim}

Now repeat this process for the same gene with the unaligned data. Can you see the difference?

\hypertarget{trimal}{%
\section{Trimal}\label{trimal}}

TBD

\hypertarget{concatenated-data-analysis}{%
\section{Concatenated data analysis}\label{concatenated-data-analysis}}

There are multiple ways to analyze data.
The first is to concatenate everything.
Make sure you are in your Terminal.

\begin{verbatim}
conda deactivate
conda activate /mnt/homes4celsrs/shared/envs/amas

python3 /mnt/homes4celsrs/shared/envs/amas/bin/AMAS.py concat -f fasta \
-d dna --out-format fasta --part-format raxml -i *FNA.fa.fasta \
-t concatenated.fasta -p partitions.txt
\end{verbatim}

Note that if some data was not found for some genes you may need to delete files (use the \texttt{rm} command) to get AMAS to concatenate your data.

\hypertarget{building-trees-with-iqtree}{%
\subsection{Building trees with IQTree}\label{building-trees-with-iqtree}}

We will build our first tree using our complete concatenated dataset.
The program IQtree infers phylogenetic trees by maximum likelihood.
This approach starts with a tree and calculates the likelihood of the data on the tree (i.e.~calculating the probability of each site fitting the tree given a model of substitution and multiplying them together).
We use the General Time Reversible (GTR) model to allow sites to change back and forth among different bases with particular probabilities.
We also allow a Gamma (G) distribution of rates of substitution across sites.
We allow partitioning of the data by gene so that different genes can evolve according to different models.
Additionally we have included bootstrapping in our analysis to get a measure of support for relationships.

\begin{verbatim}
conda deactivate
conda activate /mnt/homes4celsrs/shared/envs/iqtree

iqtree2 -nt 2 -s concatenated.fasta -spp partitions.txt -pre iqtree_tree -B 1000 -m GTR+G
\end{verbatim}

\hypertarget{viewing-trees}{%
\subsection{Viewing trees}\label{viewing-trees}}

\begin{itemize}
\tightlist
\item
  Make a new RScript to view your tree in R.
\item
  Load the libraries tidyverse and ape.
\item
  Use the \texttt{read.tree} command and provide the path to the tree as the argument
\item
  Use the plot command providing the tree variable as the argument, then add \texttt{,\ show.node.label\ =\ TRUE}
\end{itemize}

\hypertarget{rooting-your-tree}{%
\subsubsection{Rooting your tree}\label{rooting-your-tree}}

Our tree-building programs create trees that are unrooted because we do not know the direction of changes among species (e.g.~for a single difference we are unable to say if an A mutated to a T or vice versa).
Thus, we should properly view out trees as unrooted.

\begin{verbatim}
plot(tax1, "u", show.node.label = TRUE, cex = .5)
\end{verbatim}

Note that I have added an argument to make the font size bit smaller so we can read all of the labels.
You should adjust this value as needed and use the Zoom feature to better view your tree.

However, we have created taxon sets where one of the included species is known to be more distantly related.
We can use this to set the root for the tree.
You should view the list of taxa for your group in the file in the shared folder for this workshop.
The last taxon in your list is the outgroup.
You can root your tree as in the following example using the \texttt{root} command and the outgroup argument.
Note that your outgroup will be different if you are using a different taxon set.
Additionally you must use the tip label that you can see on your tree not the labels in this spreadsheet.
If you want to view the tip labels as a list use

\begin{verbatim}
tax1$tip.label
tax1_root <- root(tax1, outgroup = "A217_CKDN220062756-1A")
\end{verbatim}

Now that you have rooted your tree you can plot this new tree as before.

\hypertarget{relabeling-your-tips-with-species-information}{%
\subsubsection{Relabeling your tips with species information}\label{relabeling-your-tips-with-species-information}}

There are a couple of ways to relabel the tips of your trees with correct species names.
We will use a straightforward, manual approach.
We will look at the list of tips and then manually replace them with a list of species names.
You should keep in mind that this approach is prone to error and challenging for many taxa; however an automated approach is a bit more challenging to set up in this course.

\begin{itemize}
\tightlist
\item
  List the tip labels as you did in the previous step
\item
  Make a list of new tip labels - it should look something like the following but with more and different species
\end{itemize}

\begin{verbatim}
sp_names <- c("Centropogon mandonis", "Siphocampylus andinus", "Centropogon mandonis")
\end{verbatim}

\begin{itemize}
\tightlist
\item
  Assign these species names to the tip labels
\end{itemize}

\begin{verbatim}
tax1$tip.label <- sp_names
\end{verbatim}

If you relabeled the tips of your unrooted tree you need to root the tree with the appropriate outgroup name. You can then plot this rooted and corrected tree.

\hypertarget{tree-for-all-data-using-output-of-individual-hybpiper-runs}{%
\subsection{Tree for all data using output of individual hybpiper runs}\label{tree-for-all-data-using-output-of-individual-hybpiper-runs}}

In the prior analysis we examined all of the data combined.
For the next analysis we will estimate individual gene trees and then combine these into a species tree.

\begin{verbatim}
iqtree -s concatenated.fasta -S partitions.txt -pre iqtree.loci -nt 2
\end{verbatim}

-S tells IQ-TREE to infer separate trees for each partition.
The output files are the same, except that now your treefile will contain a set of gene trees.

For a more extensive tutorial on IQTree see
\url{http://www.iqtree.org/workshop/molevol2019}

\hypertarget{species-tree-analyses}{%
\section{Species tree analyses}\label{species-tree-analyses}}

An alternative is to use a species tree approach.
We will use the software ASTRAL.

\begin{verbatim}
java -jar /mnt/homes4celsrs/shared/ASTRAL/astral.5.7.8.jar -i iqtree.loci.treefile -o astral_output.tre 2>astral.log
\end{verbatim}

\begin{itemize}
\tightlist
\item
  -i is the flag for the input file
\item
  -o is the flag for the output file
\item
  2\textgreater{} saves the output log information
\end{itemize}

The output in is Newick format and gives:

\begin{itemize}
\item
  the species tree topology
\item
  branch lengths in coalescent units (only for internal branches or for terminal branches if that species has multiple individuals)
\item
  branch supports measured as local posterior probabilities
\item
  View this tree as before and compare it to the other tree you estimated using IQTree for partitioned concatenated data.
\end{itemize}

More information on astral can be found at
\url{https://github.com/smirarab/ASTRAL}

\hypertarget{support-for-relationships}{%
\section{Support for relationships}\label{support-for-relationships}}

In our initial IQTree analysis we obtained bootstrap support for each node.
An alternative approach are gene and site concordance factors.
For more information see \url{http://www.robertlanfear.com/blog/files/concordance_factors.html} .
You can compute gCF and sCF for the tree inferred under the partition model:

\begin{verbatim}
iqtree -t iqtree_tree.treefile --gcf iqtree.loci.treefile -s concatenated.fasta --scf 100 -nt 2
\end{verbatim}

\begin{itemize}
\item
  -t specifies a tree
\item
  --gcf specifies the gene-trees file
\item
  --scf 100 to draw 100 random quartets when computing sCF.
\item
  Repeat for the astral tree if it differs.
\item
  View the tree (iqtree\_tree.treefile.cf.tree) in R with these support values. The tree will show boostrap/gcf/scf on the nodes.
\end{itemize}

\hypertarget{mrbayes}{%
\section{MrBayes}\label{mrbayes}}

The next approach we will use for tree building is implemented in MrBayes.
We need to do some setup before we can run this software.
\texttt{R} has some tools we can use to convert out data to the write format and add some instructions.
Note that some of the following was adapted from \url{https://gtpb.github.io/MEVR16/bayes/mb_example.html}

\begin{itemize}
\tightlist
\item
  Load your data into R (use a new script for organizational purposes)
\end{itemize}

\begin{verbatim}
library(ape)
library(tidyverse)

myseqs <- read.dna("results/concatenated.fasta",format="fasta",as.matrix=FALSE)
\end{verbatim}

MrBayes requires Nexus format, with an added block giving instructions to MrBayes. We first save the data as Nexus format, and read back in to manipulate further.

\begin{verbatim}
write.nexus.data(as.character(myseqs),"results/concatenated.nex",interleaved=TRUE,gap="-",missing="N")
myseqs_nex <- readLines("results/concatenated.nex")
\end{verbatim}

Fix missing because AMAS using ? for missing

\begin{verbatim}
myseqs_nex <- gsub('\\?','n',myseqs_nex)
\end{verbatim}

First execution block

\begin{verbatim}
mbblock1 <- "
begin mrbayes;
  set autoclose=yes;
"
\end{verbatim}

Second execution block has information about partitions.
We get this from the partitions file generated by AMAS but we have to do some reformatting.

\begin{itemize}
\tightlist
\item
  Read the file with \texttt{read.table}
\item
  Get just the last column using the \texttt{pull} function
\item
  Now you can convert this list to a string to be inserted into the output file
\end{itemize}

\begin{verbatim}
partition_string <- toString(partitions,sep = ", ")
\end{verbatim}

Finally, put all the information necessary for MrBayes into a single string

\begin{verbatim}
mbblock2 <- paste0("  partition favored = ",nrow(partition_file),":",partition_list,";")
\end{verbatim}

Add a block for the MCMC parameters.

\begin{verbatim}
mbblock2 <- "
  mcmc ngen=10000000 nruns=2 nchains=2 samplefreq=1000;
  sump;
  sumt;
end;
"
\end{verbatim}

We then paste the blocks together and write to a file.

\begin{verbatim}
myseqs_nexus_withmb <- paste(paste(myseqs_nex,collapse="\n"),mbblock1,mbblock2,sep="")
write(myseqs_nexus_withmb,file="concatenated.nex.mb")
\end{verbatim}

Now run (on the command line) (probably in background as you did previously).

\begin{verbatim}
../../shared/bin/mb "results/concatenated.nex.mb"
\end{verbatim}

Before looking at your tree check the output. In some cases you may need to run your analysis for longer and a note to this effect will be toward the end of the output. Make sure to give the log file for this run a relevant name. You don't want to overwrite the output from HybPiper and you might want to keep this output if you do a longer run. Additionally, if you rerun the analysis with additional time make sure you don't overwrite the current tree output.

You can find your tree in your results folder with the extension \texttt{.con.tre}.
Because Mr.~Bayes generates output in a particular format you will need to read it in slightly differently and convert it to the standard \texttt{phylo} format.

\begin{verbatim}
mrbayes_tree <- treeio::read.mrbayes("results/concatenated.nex.mb.con.tre")
mrbayes_phylo <- treeio::as.phylo(mrbayes_tree)
probs = as_tibble(mrbayes_tree) %>% pull(prob_percent)
mrbayes_phylo$node.label = probs[(length(mrbayes_phylo$tip.label)+1):length(probs)]
\end{verbatim}

\begin{itemize}
\tightlist
\item
  Reroot your tree
\item
  Fix the names to be species
\item
  Plot your tree
\end{itemize}

\hypertarget{what-do-your-trees-tell-you}{%
\section{What do your trees tell you?}\label{what-do-your-trees-tell-you}}

\hypertarget{dates-and-evolution}{%
\chapter{Dates and Evolution}\label{dates-and-evolution}}

\hypertarget{dates-times-on-trees}{%
\section{Dates / Times on Trees}\label{dates-times-on-trees}}

If you look at your tree, you will notice the tips don't ``line up''
on the right side.
It's as if not all species made it to the present day.
This is only an appearance because branch lengths are in substitutions per site and not
all species have the same substitution rate.
We can adjust the branch lengths to match time using the rates.
Usually we have calibration dates with complex probability distributions.
In this case we are taking an extremely simple approach solely for the purpose of seeing
a phylogeny where the branches appear to be time.

\begin{verbatim}
mrbayes_chronos <- chronos(mrbayes_phylo)
\end{verbatim}

We can also generate the tree using a ``strict clock model''.

\begin{verbatim}
### strict clock model:
clock_rate <- chronos.control(nb.rate.cat = 1)
mrbayes_chronos_clock <- chronos(mrbayes_phylo, model = "discrete", control = ctrl)
\end{verbatim}

Now let's plot all three trees together so you can compare the branch lengths.

\begin{verbatim}
par(mfrow = c(3,1))
plot(mrbayes_phylo, cex = 1)
plot(mrbayes_chronos, cex = 1)
plot(mrbayes_chronos_clock, cex = 1)
\end{verbatim}

\hypertarget{understanding-the-likelihood-function}{%
\chapter{Understanding the likelihood function}\label{understanding-the-likelihood-function}}

\hypertarget{background-1}{%
\section{Background}\label{background-1}}

Ari Martinez spent multiple summers doing field work in the Peruvian Amazonian. In 2011, Ari observed an interesting behavior in heterospecific flocks. Depending on where on the forest a bird was feeding its response would differ. For example if a bird is feeding on the ground(dead-leaf gleaning) then the bird would ``freeze'' for some time, but most of the flycatchers that feed on the top of the trees wouldn't even bother to stop.

Ari started timing the freezing behavior for some of these birds and was able to collect a small sample. He would play an alarm call and then measure the time birds were freezing depending on where they were foraging.

This is his sample and you will be using it to understand evidence in likelihood.

\begin{Shaded}
\begin{Highlighting}[]
\CommentTok{\#Set you workind directory}
\FunctionTok{setwd}\NormalTok{(}\StringTok{\textquotesingle{}../../shared/AndesWorkshop2023/\textquotesingle{}}\NormalTok{)}
\NormalTok{bird.alarms}\OtherTok{\textless{}{-}}\FunctionTok{read.csv}\NormalTok{(}\StringTok{"birdalarms.csv"}\NormalTok{,}\AttributeTok{stringsAsFactors=}\ConstantTok{FALSE}\NormalTok{)}
\FunctionTok{head}\NormalTok{(bird.alarms)}
\end{Highlighting}
\end{Shaded}

\hypertarget{the-model}{%
\subsection{The model}\label{the-model}}

We are going to model freezing time using an exponential distribution with parameter \(1/\theta\), where \(\theta\) is the expected time that birds of a group freeze.

What is an expoential distribution with parameter \(1/\theta\)?

\hypertarget{the-likelihood-function}{%
\subsection{The likelihood function}\label{the-likelihood-function}}

The likelihood function \(L(\theta;X)=P(\theta|X)\) is the product of the exponential density evaluated in every value of the sample observed. In R we can code it using the \texttt{dexp()} probability function

\begin{Shaded}
\begin{Highlighting}[]
\NormalTok{likelihood.function}\OtherTok{\textless{}{-}} \ControlFlowTok{function}\NormalTok{(parameter, observations)\{}
\NormalTok{    probabilities }\OtherTok{\textless{}{-}}\FunctionTok{dexp}\NormalTok{(observations, }\AttributeTok{rate=}\DecValTok{1}\SpecialCharTok{/}\NormalTok{parameter,}\AttributeTok{log=}\ConstantTok{FALSE}\NormalTok{)}
\NormalTok{    L }\OtherTok{\textless{}{-}} \FunctionTok{prod}\NormalTok{(probabilities)}
    \FunctionTok{return}\NormalTok{(}\SpecialCharTok{{-}}\NormalTok{L)}
\NormalTok{\}}
\end{Highlighting}
\end{Shaded}

The likelihood function is much more difficult depending on the model we select. We will discuss more about its construction when we discuss substitution models. However the way to think about it is alwasy as the probability of the data given a model (or set of parameters) \(P(Daata|Model)\).

Knowing this definition of the likelihood then let's ask
+ What is the input of the likelihood function?
+ What is the output of the likelihood function? ( and why is it negative?)

\hypertarget{a-mini-example-with-a-sample-size-of-two}{%
\subsection{A mini example with a sample size of two}\label{a-mini-example-with-a-sample-size-of-two}}

If we have a sample size of two birds freezing responses \(X=2,10\) the maximum likelihood estimate is the average \(\hat{\theta}=\sum_{i=1}^n x_i/n\). Therefore

\begin{Shaded}
\begin{Highlighting}[]
\NormalTok{(mle}\OtherTok{\textless{}{-}}\NormalTok{(}\DecValTok{2}\SpecialCharTok{+}\DecValTok{10}\NormalTok{)}\SpecialCharTok{/}\DecValTok{2}\NormalTok{)}
\end{Highlighting}
\end{Shaded}

However, most of the time in difficult likelihoods it is impossible to calculate exactly who the mle so we do it numerically

\begin{Shaded}
\begin{Highlighting}[]
\NormalTok{(likelihood.optimization}\OtherTok{\textless{}{-}}\FunctionTok{optimize}\NormalTok{(}\AttributeTok{f=}\NormalTok{likelihood.function, }\AttributeTok{interval=}\FunctionTok{c}\NormalTok{(}\DecValTok{0}\NormalTok{,}\DecValTok{10}\NormalTok{), }\AttributeTok{observations=}\FunctionTok{c}\NormalTok{(}\DecValTok{2}\NormalTok{,}\DecValTok{10}\NormalTok{)))}
\end{Highlighting}
\end{Shaded}

The function \texttt{optimize} is used when we have unidimensional functions (one parameter). For multidimensional we use \texttt{optim} or even better \texttt{nloptr} from the package with the same name.

What are the outputs?

\hypertarget{the-bird-alarm-example}{%
\subsection{The bird alarm example}\label{the-bird-alarm-example}}

In Ari's example we have the dead-leaf gleaning species

\begin{Shaded}
\begin{Highlighting}[]
\NormalTok{dl.forager}\OtherTok{\textless{}{-}}\NormalTok{ bird.alarms}\SpecialCharTok{$}\NormalTok{Response[}\FunctionTok{which}\NormalTok{(bird.alarms}\SpecialCharTok{$}\NormalTok{Forage}\SpecialCharTok{==}\StringTok{"DL"}\NormalTok{)] }
\end{Highlighting}
\end{Shaded}

and the flycatchers

\begin{Shaded}
\begin{Highlighting}[]
\NormalTok{f.forager}\OtherTok{\textless{}{-}}\NormalTok{bird.alarms}\SpecialCharTok{$}\NormalTok{Response[}\FunctionTok{which}\NormalTok{(bird.alarms}\SpecialCharTok{$}\NormalTok{Forage}\SpecialCharTok{==}\StringTok{"F"}\NormalTok{)] }\CommentTok{\#11}
\end{Highlighting}
\end{Shaded}

\hypertarget{calculate-the-mle-for-dl-and-f-and-the-likelihood-value-evaluated-at-the-mle}{%
\subsubsection{Calculate the MLE for DL and F and the likelihood value evaluated at the MLE}\label{calculate-the-mle-for-dl-and-f-and-the-likelihood-value-evaluated-at-the-mle}}

You should be getting approximately the following values:

\begin{Shaded}
\begin{Highlighting}[]
\NormalTok{dl.optimization}\OtherTok{\textless{}{-}}\FunctionTok{optimize}\NormalTok{(}\AttributeTok{f=}\NormalTok{likelihood.function, }\AttributeTok{interval=}\FunctionTok{c}\NormalTok{(}\DecValTok{10}\NormalTok{,}\DecValTok{30}\NormalTok{), }\AttributeTok{observations=}\NormalTok{dl.forager)}

\NormalTok{(mle.dl}\OtherTok{\textless{}{-}}\NormalTok{dl.optimization}\SpecialCharTok{$}\NormalTok{minimum)}
\NormalTok{(likelihoodval.dl}\OtherTok{\textless{}{-}} \SpecialCharTok{{-}}\NormalTok{dl.optimization}\SpecialCharTok{$}\NormalTok{objective)}

\NormalTok{(mle.f}\OtherTok{\textless{}{-}}\FunctionTok{mean}\NormalTok{(f.forager))}
\NormalTok{(likelihoodval.f}\OtherTok{\textless{}{-}}\SpecialCharTok{{-}}\FunctionTok{likelihood.function}\NormalTok{(mle.f,}\AttributeTok{observations=}\NormalTok{f.forager))}
\end{Highlighting}
\end{Shaded}

So, are the flycatchers behaving differently than the dead-leaf gleaners? What is the evidence? Can we compare these likelihoods or MLEs?

\hypertarget{where-is-the-evidence-for-different-behavior}{%
\subsection{Where is the evidence for different behavior?}\label{where-is-the-evidence-for-different-behavior}}

Likelihood functions are not only about the maximum likelihood estimate. Likelihoods represent plausibility. This is represented in the full likelihood function so it is important to explore it.

We select a series of parameters and measure their plausibility for example in the interval \((0,50)\)

\begin{Shaded}
\begin{Highlighting}[]
\NormalTok{parameter.vals}\OtherTok{\textless{}{-}}\FunctionTok{seq}\NormalTok{(}\FloatTok{0.0001}\NormalTok{,}\DecValTok{50}\NormalTok{,}\FloatTok{0.01}\NormalTok{) }\CommentTok{\#creating an interval for possible values for the likelihood}
\NormalTok{long}\OtherTok{\textless{}{-}}\FunctionTok{length}\NormalTok{(parameter.vals) }
\CommentTok{\# Evaluating the likelihood for each of those values}
\NormalTok{p.likelihoodf}\OtherTok{\textless{}{-}}\FunctionTok{rep}\NormalTok{(}\DecValTok{0}\NormalTok{,long)}
\ControlFlowTok{for}\NormalTok{ (i }\ControlFlowTok{in} \DecValTok{1}\SpecialCharTok{:}\NormalTok{long)\{}
\NormalTok{p.likelihoodf[i]}\OtherTok{\textless{}{-}} \SpecialCharTok{{-}}\FunctionTok{likelihood.function}\NormalTok{(parameter.vals[i],}\AttributeTok{observations=}\NormalTok{f.forager) }\CommentTok{\# Remeber is negative so we need to add a sign}
\NormalTok{\}}
\end{Highlighting}
\end{Shaded}

In different studies likelihood can be represented using different scales

\begin{Shaded}
\begin{Highlighting}[]
\FunctionTok{par}\NormalTok{(}\AttributeTok{mfrow=}\FunctionTok{c}\NormalTok{(}\DecValTok{1}\NormalTok{,}\DecValTok{3}\NormalTok{))}
\CommentTok{\# Straight likelihood function}
\FunctionTok{plot}\NormalTok{(parameter.vals, p.likelihoodf, }\AttributeTok{type=}\StringTok{"l"}\NormalTok{,}\AttributeTok{main=}\StringTok{"Likelihood for flycatchers"}\NormalTok{,}\AttributeTok{xlab=}\FunctionTok{expression}\NormalTok{(theta),}\AttributeTok{ylab=}\StringTok{"Likelihood"}\NormalTok{, }\AttributeTok{lwd=}\DecValTok{2}\NormalTok{,}\AttributeTok{xlim=}\FunctionTok{c}\NormalTok{(}\DecValTok{0}\NormalTok{,}\DecValTok{5}\NormalTok{))}


\CommentTok{\#log{-}likelihood function, most commonly used}
\FunctionTok{plot}\NormalTok{(parameter.vals, }\FunctionTok{log}\NormalTok{(p.likelihoodf), }\AttributeTok{type=}\StringTok{"l"}\NormalTok{,}\AttributeTok{main=}\StringTok{"log{-}likelihood for flycatchers"}\NormalTok{,}\AttributeTok{xlab=}\FunctionTok{expression}\NormalTok{(theta),}\AttributeTok{ylab=}\StringTok{"Likelihood"}\NormalTok{,}\AttributeTok{lwd=}\DecValTok{2}\NormalTok{,}\AttributeTok{xlim=}\FunctionTok{c}\NormalTok{(}\DecValTok{0}\NormalTok{,}\DecValTok{5}\NormalTok{))}

\CommentTok{\# Relative likelihood: Likelihood divided by the likelihood value at the MLE}
\FunctionTok{plot}\NormalTok{(parameter.vals, p.likelihoodf}\SpecialCharTok{/}\NormalTok{likelihoodval.f, }\AttributeTok{type=}\StringTok{"l"}\NormalTok{,}\AttributeTok{main=}\StringTok{"Relative likelihood for flycatchers"}\NormalTok{,}\AttributeTok{xlab=}\FunctionTok{expression}\NormalTok{(theta),}\AttributeTok{ylab=}\StringTok{"Likelihood"}\NormalTok{,}\AttributeTok{lwd=}\DecValTok{2}\NormalTok{,}\AttributeTok{xlim=}\FunctionTok{c}\NormalTok{(}\DecValTok{0}\NormalTok{,}\DecValTok{5}\NormalTok{))}
\end{Highlighting}
\end{Shaded}

What do you think about the sample size for flycatchers?

\hypertarget{plot-the-likelihood-for-dead-leaf-gleaners}{%
\subsubsection{Plot the likelihood for dead-leaf gleaners}\label{plot-the-likelihood-for-dead-leaf-gleaners}}

It should look like this

\begin{Shaded}
\begin{Highlighting}[]
\NormalTok{p.likelihooddl}\OtherTok{\textless{}{-}}\FunctionTok{rep}\NormalTok{(}\DecValTok{0}\NormalTok{,long)}
\ControlFlowTok{for}\NormalTok{ (i }\ControlFlowTok{in} \DecValTok{1}\SpecialCharTok{:}\NormalTok{long)\{}
\NormalTok{p.likelihooddl[i]}\OtherTok{\textless{}{-}} \SpecialCharTok{{-}}\FunctionTok{likelihood.function}\NormalTok{(parameter.vals[i],}\AttributeTok{observations=}\NormalTok{dl.forager)}
\NormalTok{\}}

\FunctionTok{par}\NormalTok{(}\AttributeTok{mfrow=}\FunctionTok{c}\NormalTok{(}\DecValTok{1}\NormalTok{,}\DecValTok{3}\NormalTok{))}
\FunctionTok{plot}\NormalTok{(parameter.vals, p.likelihooddl, }\AttributeTok{type=}\StringTok{"l"}\NormalTok{,}\AttributeTok{main=}\StringTok{"Likelihood for flycatchers"}\NormalTok{,}\AttributeTok{xlab=}\StringTok{"Rate Parameter"}\NormalTok{,}\AttributeTok{ylab=}\StringTok{"Likelihood"}\NormalTok{,}\AttributeTok{lty=}\DecValTok{2}\NormalTok{,}\AttributeTok{col=}\StringTok{"red"}\NormalTok{,}\AttributeTok{lwd=}\DecValTok{2}\NormalTok{)}

\FunctionTok{plot}\NormalTok{(parameter.vals, }\FunctionTok{log}\NormalTok{(p.likelihooddl), }\AttributeTok{type=}\StringTok{"l"}\NormalTok{,}\AttributeTok{main=}\StringTok{"log{-}likelihood for flycatchers"}\NormalTok{,}\AttributeTok{xlab=}\StringTok{"Rate Parameter"}\NormalTok{,}\AttributeTok{ylab=}\StringTok{"Likelihood"}\NormalTok{,}\AttributeTok{lty=}\DecValTok{2}\NormalTok{,}\AttributeTok{col=}\StringTok{"red"}\NormalTok{,}\AttributeTok{lwd=}\DecValTok{2}\NormalTok{)}

\FunctionTok{plot}\NormalTok{(parameter.vals, p.likelihooddl}\SpecialCharTok{/}\NormalTok{likelihoodval.dl, }\AttributeTok{type=}\StringTok{"l"}\NormalTok{,}\AttributeTok{main=}\StringTok{"Relative likelihood for flycatchers"}\NormalTok{,}\AttributeTok{xlab=}\StringTok{"Rate Parameter"}\NormalTok{,}\AttributeTok{ylab=}\StringTok{"Likelihood"}\NormalTok{,}\AttributeTok{lty=}\DecValTok{2}\NormalTok{,}\AttributeTok{col=}\StringTok{"red"}\NormalTok{,}\AttributeTok{lwd=}\DecValTok{2}\NormalTok{)}
\end{Highlighting}
\end{Shaded}

\hypertarget{are-the-freezing-times-different-for-the-two-groups}{%
\subsection{Are the freezing times different for the two groups?}\label{are-the-freezing-times-different-for-the-two-groups}}

Usually you will go ahead and do some statistical test to say ``I reject that the average freezing time of the groups is the same''. Except that you can't do a T-test (not normal), sample sizes are really small (so not so much power). The evidence of likelihood comes to the rescue.

Using relative likelihoods we can compare the evidence between the two groups

\begin{Shaded}
\begin{Highlighting}[]
\FunctionTok{plot}\NormalTok{(parameter.vals, p.likelihoodf}\SpecialCharTok{/}\NormalTok{likelihoodval.f, }\AttributeTok{type=}\StringTok{"l"}\NormalTok{,}\AttributeTok{main=}\StringTok{"Evidence for responses"}\NormalTok{,}\AttributeTok{xlab=}\StringTok{"Rate Parameter"}\NormalTok{,}\AttributeTok{ylab=}\StringTok{"Likelihood"}\NormalTok{)}
\FunctionTok{lines}\NormalTok{(parameter.vals, p.likelihooddl}\SpecialCharTok{/}\NormalTok{likelihoodval.dl,}\AttributeTok{lty=}\DecValTok{2}\NormalTok{,}\AttributeTok{col=}\StringTok{"red"}\NormalTok{,}\AttributeTok{lwd=}\DecValTok{2}\NormalTok{)}
\FunctionTok{legend}\NormalTok{(}\AttributeTok{x=}\DecValTok{35}\NormalTok{,}\AttributeTok{y=}\FloatTok{0.8}\NormalTok{, }\AttributeTok{col=}\FunctionTok{c}\NormalTok{(}\StringTok{"black"}\NormalTok{,}\StringTok{"red"}\NormalTok{),}\AttributeTok{legend=}\FunctionTok{c}\NormalTok{(}\StringTok{"flycatcher"}\NormalTok{,}\StringTok{"dead{-}leaf"}\NormalTok{),}\AttributeTok{lty=}\DecValTok{1}\SpecialCharTok{:}\DecValTok{2}\NormalTok{)}
\end{Highlighting}
\end{Shaded}


  \bibliography{book.bib,packages.bib}

\end{document}
